\documentclass{article}

\usepackage{graphics}
\usepackage{graphicx}
\usepackage{epsfig}   % Enable PS and EPS graphics in latex
\usepackage{placeins} %Package to  prevent floats from crossing a barrier (green book)
\usepackage{natbib}   % Enable insertion of cites in text

\renewcommand\contentsname{your title for TOC}
\renewcommand\listfigurename{your title for LoF}
\renewcommand\listtablename{your title for LoT}
\renewcommand\refname{your title for Bib.}
\renewcommand\abstractname{your title for abstract}

%\usepackage{caption}
%\captionsetup[table]{name=Tabla}
%\usepackage[spanish]{babel}



%%%%%%%%%%%%%%%%%%%%%%%%%%%%%%%%%%%%%%%%%%%%%%%%%%%%%%%%%%%%%%%%%%%%%%%%
\setlength{\voffset}{-1in}          % Reset the top margin back to zero
\setlength{\hoffset}{-1in}          % Reset the left margin to zero

\setlength{\oddsidemargin}{2.8cm}   % Odd side left margin to 30mm
\setlength{\evensidemargin}{2.8cm}  % Even side left margin to 30mm
\setlength{\textwidth}{15.4cm}  % Text width to be 15.9cm (an letter page
                                % is 15.4cm+3cm+3cm aproximadamente = 21.5cm wide)
\setlength{\topmargin}{3.7cm}       % Top margin
\setlength{\textheight}{20cm} % Text height to be 21.7cm (an letter
                                % page is 20cm+3.7cm+2.5cm aproximadamente = 27.9cm long)

%\usepackage{calligra}
%\usepackage[T1]{fontenc}


\begin{document}
%\calligra

\title{Un documento sin ortografia \\ Y con titulo en dos lineas\thanks{Gracias a mis patrocinadores}}
\author{Unautor sinortografia\thanks{Gracias a mi abuelita}\\ Universidad Perdida \and Uncoautor sincerebro}
\date{Primera version Agosto 26. Ultima version \today}

\maketitle

\begin{abstract}
    Nunca aprendi ortografia, pero estoy aprendiendo. Y el resumen esta en ingles y no en espanol.
\end{abstract}

\newpage
    \tableofcontents
    \listoffigures
    \listoftables

\newpage
\section{Introduccion}\label{sec.intro}
    Siempre quise aprender a escribir. No me gustan las margenes de este documento.

    \begin{table}[t]
        \centering
        \caption[Nr of news by section and authored]{Nr of news by section and authored}
        \begin{tabular}{lrr}
        \hline
        Newspaper section & No. of news & Nr. of authored news \\
        Economy & 1,015 & 243 \\
        Editorial - Opinion & 439 & 288 \\
        \hline
        \multicolumn{3}{p{0.70\textwidth}}{\footnotesize{\emph{Note}: Authored news correspond to those where the name of author is published. News without author are assumed to be authored by newspaper staff. Editorial - opinion section include editorials and op-eds, op-eds are those which include the name of the author.}}\\
        \end{tabular}
        \label{tab:news.description}
    \end{table}

\subsection{Explicacion de la introduccion}\label{ssec.intro.explicacion}
    Pero nunca aprendi.

    \begin{figure}[t]
        \centering
            \includegraphics[width=0.5\textwidth,angle=90]
                {c:/rodrigo/proyecto_cursolatex/text/figures/avion.eps}
                \caption{Un avi\'on.}
        \label{fig:figura}
    \end{figure}

\FloatBarrier
\subsubsection{Explicacion de la introduccion}\label{ssec.intro.explicacion}
    Pero nunca aprendi.

\paragraph{Explicacion de la introduccion}\label{ssec.intro.explicacion}
    Pero nunca aprendi.

\subparagraph{Explicacion de la introduccion}\label{ssec.intro.explicacion}
    Pero nunca aprendi. Mejor busco a \citet{echavarria2010}

\section{Conclusion}
    Mejor no escribo mas como en seccion \ref{sec.intro} y subseccion \ref{ssec.intro.explicacion}.

\bibliographystyle{c:/rodrigo/latex/bibstyles/chicago} %Path and file of bibliography style
\bibliography{c:/rodrigo/latex/rotabori}

\end{document}
